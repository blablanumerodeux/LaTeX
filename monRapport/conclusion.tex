\chapter*{Conclusion}
\addcontentsline{toc}{chapter}{Conclusion}

Tout au long de ce rapport j'ai abordé de nombreux points sur mes différentes missions. J'ai évoqué les problèmes techniques qui m'ont été posés. J'ai parler des compétences qu'il m'a fallu acquérir, mais j'ai très peu évoqué l'aspect humain de cette année. Pourtant, ce n'est pas un aspect négligeable de la vie en entreprise. L'alternance m'a permis de grandir aussi bien du côté humain que du côté technique. Le rythme très particulier de l'alternance m'a mis à l'épreuve socialement. 

D'un point de vu professionnel l'apprentissage m'a apporté l'expérience qui me permettra de me démarquer des autres sur le marché du travail. L'entreprise m'a apporté des connaissances que je n'aurais jamais pu acquérir à l'IUT. J'ai travaillé sur du matériel très spécifique et très sécurisé. Je me suis également rendu compte de la difficulté du travail de manager. 

Tous les problèmes que j'ai eu se sont révélés être enrichissants. J'ai finalement appris beaucoup sur le monde du travail. L'expérience que j'ai acquise me permettra de me rapprocher de mon projet de carrière dans le développement bas niveau. Enfin les connaissances que j'ai également acquises à l'IUT sont plus que bénéfiques. Elles sont un vrai pilier pour mon avenir. Et c'est pourquoi je serais heureux de continuer mes études en apprentissage.

% parler de l'iut
% posibilites d'evolutions de ma cariere 
% ce que j'aurais pu/voulu faire
% negatif puis positif
% ouverture sur mon diplome d'ingenieur.
