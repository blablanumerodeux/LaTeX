 \begin{changemargin}{-10mm}{-5mm}

\begin{minipage}[c][150pt][c]{.40\linewidth} %[hauteur][position]{largeur}
	{\normalsize 
		\begin{flushleft}
		\vspace{3mm}
	      	CERN\\
		Organisation Europ�enne pour\\
		la Recherche Nucl�aire\\
		F-01631 CERN Cedex\\
		France\\
		\url{http://www.cern.ch/}\\
		\vspace{2mm}
		\includegraphics[width=2cm]{./img/CERN_logo}
		\vspace{3mm}
		\end{flushleft}
	}
   \end{minipage}\hfill
%}
\hspace{15mm}
%
%\Ovalbox{
   \begin{minipage}[c][150pt][c]{.40\linewidth} %[hauteur][position]{largeur}
	{\normalsize 
		\begin{flushright}
		\vspace{3mm}
		K�vin Raymond\\
		IUT Annecy\\
		D�partement GEII\\
		Ann�e 2007\\
		\vspace{2mm}
		\includegraphics[width=5cm]{./img/A}
		\vspace{3mm}
		\end{flushright}
	}
   \end{minipage}
%}
\vspace{15mm}

\begin{center}
\textbf{\large{R�sum�}}
\end{center}

\thispagestyle{empty}
\begin{center}
\textbf{Mesure d'une temp�rature avec pr�cision et transmission par liaison sans fil}
\end{center}




%\fontsize{11pt}{1}\selectfont

Dans cette �tude, nous abordons la mesure d'une temp�rature ambiante 
avec pr�cision. Apr�s avoir d�crit le fonctionnement globale d'une 
telle mesure et des diff�rentes possibilit�s de mises en \oe uvre, 
nous �tudierons le d�faut de la cha�ne de mesure et l'incertitude introduite.

Ceci passe par l'utilisation des logiciels libres de Conception �lectronique 
Assist� par Ordinateur et de programation des microcontr�leurs, ainsi que 
par l'utilisation de divers protocoles comme le ZigBee ou l'USB.

Pour une plus grande portabilit�, l'�tude est r�alis�e dans un environnement GNU/Linux.
\vspace{0.25cm}

\noindent\textbf{Mots-cl�s~:} Mesure de temp�rature, RTD, PRTD, Pt100, ZigBee, Xbee, USB, PIC, GNU, Linux. %\fontsize{12pt}{1.5}\selectfont


\vspace{10mm}
\begin{center}
	\underline{\hspace{3cm}} % Trace une fine ligne de 3 cm...
\end{center}
\vspace{10mm}
%\vskip15mm
\selectlanguage{english}
\begin{center}
\textbf{\large{Abstract}}
\end{center}
\begin{center}
\textbf{Mesure of high precision temperature and wireless equipment}
\end{center}

%\vskip5mm



%\fontsize{10pt}{1}\selectfont

In this study, we move on the precision sensor temperature. After a global description of some possibilities
of exploitation, we will study the default of the circuit conditioning.

Getting started with some open source software like PIC programmation's and Electronic Design Automation tools.
With some new protocoles like USB and Zigbee solutions.

Of course, this study is based with a GNU/Linux computeur.


\vspace{0.25cm}
\noindent\textbf{Keywords~:} Temperature sensor, RTD, PRTD, Pt100, ZigBee, Xbee, USB, PIC, GNU, Linux.
\fontsize{12pt}{1.5}\selectfont


\vspace{13mm}
\selectlanguage{french}
\begin{center}
		Rapport disponible~: \url{http://kevin.raymond.free.fr/Stage/rapport/RapportKR.pdf}

\end{center}
\end{changemargin}

