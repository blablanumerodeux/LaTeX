\section{Programmation des PIC sous linux}
MPLAB, l'environnement distribu� par Microchip, est disponible sous Windows. Pour linux, la possibilit� est d'utiliser Wine, mais d'aucune utilit� apparente. Il est pr�f�rable d'utiliser les logiciels libres quand ceux-ci sont performants.

C'est le cas de Piklab, auquel on int�gre le compilateur, le cr�ateur de lien, le simulateur, le debugger, etc.

\subsection{installation}
\' Egalement disponible dans la plupart des d�pots, la derni�re version est � t�l�charger sur \url{http://piklab.sourceforge.net/download.php}.

Pour la programmation des PIC, le langage utilis� pour ce projet est le C. Small Device C compiler est un outil libre et performant. \' Egalement disponible sous Windows, il permet de partager plus facilement les sources et donc de trouver des exemples utilisables.

La derni�re version de SDCC est disponible l�~: \url{http://sdcc.sourceforge.net/index.php#Download}.\newline

Le programmateur utilis� est le In Circuit Debugger 2 de chez Microchip. Il permet en plus de la programmation par port s�rie ou USB d'utiliser la fonction debugger.

Pour l'utiliser il faut installer un package suppl�mentaire (icd2prog). Une version est disponible ici~: \url{http://kevin.raymond.free.fr/Stage/icd2prog-0.3.0.tar.gz}. Suivant les d�pendances, il peut ne pas �tre reconnu. Une simple installation de MPLAB sous Wine r�glera le probl�me.

%%
\subsection{utilisation}
Sous Piklab, lors de la cr�ation d'un nouveau projet, il faut sp�cifier le compilateur et le programmateur. Par d�faut beaucoup sont propos�s. Nous utiliserons SDCC avec l'ICD2.

Pour l'ICD2 il ne faut pas activer l'option "Low voltage programmation". \newline

Si tout ce passe bien, une fois connect� un rapport avec les diff�rentes tensions est affich�. Exemple~:


  \noindent Vpp du programmateur = 12.4613 V\newline
  \noindent Vdd de la cible = 5.00224 V\newline
  \noindent Vpp de la cible = 12.4613 V\newline

Etant tr�s intuitif, cet environnement est tr�s rapide � prendre en main.

